\documentclass[line,margin]{res} 

\usepackage[none]{hyphenat}

\begin{document}
	\name{David S. McDermott}
	\address{235 S. Buckhout St. State College, PA 16801}
	\address{Cellphone: (484) 904-2099}
	
	\begin{resume}
		
		\section{HIGHLIGHTS} Schreyer Scholar at The Pennsylvania State University pursuing dual degrees in computer and electrical engineering with concentrations in computer architecture and solid state engineering.  Experienced in automated data processing, visualization, and VLSI in a professional setting and operating systems programming, silicon device fabrication, and computer architecture in an academic setting. 
		
		\section{SKILLS}
		\textbf{Programming Languages:} Java, C, C++, x86 Assembly, MIPS Assembly\\
		\textbf{Scripting Languages:} Python, MATLAB, Shell (Bash), Perl, PHP\\
		\textbf{Development Tools:} GCC, GDB, Make, Visual Studio, Vim, MPLAB, Virtuoso\\
		\textbf{Design Tools:} Cadence Virtuoso, Autodesk  EAGLE, NI Multisim\\
		\textbf{Databases:} DB2, SQL Server, MySQL, MariaDB, SQLite, Neo4j
		
		\section{EDUCATION}{\sl Bachelor of Science} Computer Engineering, Electrical Engineering \\
		The Pennsylvania State University, University Park, PA \hfill May 2019\\
		College of Engineering \& Schreyer Honors College\\
		Majors: Computer Engineering, Electrical Engineering \hfill GPA: 3.53/4.00
		
		\section{EXPERIENCE}{\sl Design for Test and Characterization Intern} \hfill May 2018 - Present\\
		IBM, z/Systems, Poughkeepsie, NY
		\begin{itemize}  \itemsep -2pt
			\item Assisted in migration from single use Perl scripts to object oriented Python
			\item Assisted in development of new test platform with with Linux drivers and FPGA
			\item Developed CP characterization routines that ran 40\% faster than before
			\item Developed interactive testing and visualization platform for characterization
			\item Identified potential 3\% reduction in dynamic power of latches for z/CP
			\vspace*{-\baselineskip}		
		\end{itemize}
		\textbf{Relevant Skills:} Python, Jupyter Notebooks, SQL, Neo4j, Cypher Query, VLSI
		
		{\sl Visualization Intern} \hfill May 2017 - May 2018 \\
		The Pennsylvania State University Applied Research Laboratories, Synthetic Environments and Applications Laboratory, University Park, PA
		\begin{itemize}  \itemsep -2pt
			\item Supported development for existing visualization applications in an Agile team
			\item Parallelized physics simulation and rendering engines on existing application
			\item Developed several new data pipelines for databases, blockchains and Excel
			\vspace*{-\baselineskip}		
		\end{itemize}
		\textbf{Relevant Skills:} Java, OpenGL, C, SQL, SQLite, Apache Tomcat, Unity 3D, C\#
		\textbf{Other Information:} Active US Secret Clearance
		
		\section{PROJECTS}
		{\sl Wristband Scanner} \hfill Spring 2018 - Present\\
		The Pennsylvania State University, HackPSU
		\begin{itemize}  \itemsep -2pt
			\item Worked with a team of several other students to develop an IOT scanner
			\item Developed C++ abstractions for Arduino to interface with RFID scanner chips
			\item Worked on developing architecture for networking with web server and local proxy
		\end{itemize}
		\vspace*{-\baselineskip}		
		\textbf{Relevant Skills:} C++, Arduino, PCB Design, Antenna Design
		
		{\sl Pixel Based Texture Synthesis} \hfill Fall 2018\\
		The Pennsylvania State University, CMPSC 458
		\begin{itemize}  \itemsep -2pt
			\item Developed Python interface for texture synthesis algorithm and calling script
			\item Implemented pixel based texture synthesis using Efros-Leung method
			\item Made performance improvements using Python list comprehensions and multithreading
			\vspace*{-\baselineskip}		
		\end{itemize}
		\textbf{Relevant Skills:} Python, Computer Graphics, Statistics, Digital Image Processing

	\end{resume}
\end{document}